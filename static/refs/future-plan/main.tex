\chapter{Conclusion}
\section{Summary}
This project explored the integration of knowledge graphs with Large Language Models (LLMs) to enhance their factual accuracy and knowledge retrieval capabilities. We analyzed existing retrieval-augmented generation (RAG) techniques, including Text-Based RAG and Knowledge Graph-Based RAG (KG-RAG). To overcome limitations of these methods, we proposed KEALLM - Knowledge Graph Embedding Augmented Large Language Model. Unlike other RAG approaches, KEALLM directly incorporates knowledge graph embeddings into the LLM architecture, enabling more efficient and integrated knowledge utilization. Experiments on one-hop question answering tasks demonstrated that KEALLM consistently outperforms standard LLMs and Text-Based RAG, achieving comparable performance to KG-RAG while offering a more streamlined architecture.
\section{Future Work}
Future research will focus on expanding and refining the KEALLM architecture to further improve its performance and applicability:
\begin{itemize}
\item Adapting to More Complex Question Answering: Extend KEALLM to handle multi-hop question answering, which requires reasoning over multiple relationships in the knowledge graph. This could involve incorporating graph neural networks or other reasoning mechanisms
\item Dynamic Knowledge Integration: Explore ways to dynamically update the knowledge graph embedded in KEALLM. This would allow the model to adapt to new information and reflect changes in the real world, improving the model's accuracy and relevance
\item Evaluating KEALLM on Diverse NLP Tasks: Evaluate KEALLM's performance on a wider range of NLP tasks beyond question answering, such as summarization, text generation, and dialogue systems. This will help to assess the model's generalizability and its potential in various domains
\item Exploring Different Knowledge Graph Embedding Techniques: Experiment with alternative knowledge graph embedding techniques, such as those based on graph neural networks or complex relation representation models. This could lead to richer and more nuanced knowledge representations within KEALLM
\end{itemize}


