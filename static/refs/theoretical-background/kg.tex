\section {Knowledge graph}
Knowledge graphs (KGs) store structured knowledge as a collection of triples $KG = \{(h, r, t) \subseteq E \times R \times E\}$, where E and R respectively denote the set of entities and relations \cite{9416312}. There are four types of existing KGs based on the stored information: Encyclopedic KGs, Commonsense KGs, Domain-specific KGs, and multi-modal KGs.\\\\
\textbf{Encyclopedic KGs}  are the most common type of KGs that store general knowledge about the world. They are usually created by integrating information from various sources such as human experts, encyclopedias, and databases. Wikidata is one of the most popular encyclopedic KGs, which contains a wide range of knowledge extracted from Wikipedia articles \cite{10.1145/2629489}.\\\\
\textbf{Commonsense KGs} are designed to represent knowledge about everyday concepts such as objects and events, as well as their relationships. Unlike encyclopedic knowledge graphs, commonsense knowledge graphs often model tacit knowledge extracted from text, such as the relationship between “Car”, “UsedFor”, and “Drive”. ConceptNet \cite{speer2017conceptnet}, one of the most popular KGs, is a commonsense knowledge graph that contains a wide range of concepts and relations, which can help computers understand the meanings of words people use. \\\\
\textbf{Domain-specific KGs} are created to represent knowledge in a particular domain, such as medicine, biology, or finance. These knowledge graphs are often smaller in size than encyclopedic knowledge graphs, but they are more accurate and reliable. As an instance of a domain-specific knowledge graph, the Unified Medical Language System (UMLS) is a knowledge graph that contains biomedical concepts and their relationships \cite{Bodenreider2004TheUM}.\\\\
\textbf{Multi-modal KGs} are different from conventional knowledge graphs in that they represent facts in multiple modalities such as images, sounds, and videos. For instance, IMGpedia \cite{10.1007/978-3-319-68204-4_8}, MMKG \cite{liu2019mmkg}, and Richpedia \cite{wang2020richpedia} are examples of multi-modal knowledge graphs that incorporate both text and image information into the knowledge graphs. 