\section{Problem statement}
In the ever-evolving world of technology, the concept of an e-commerce website employing a recommender system to customize the user experience is not unfamiliar, particularly in the gadget and computer sector. Despite the prevalence of e-commerce platforms as standard marketing and sales channels for these firms, there is a need to improve the current methods of user interaction, specifically in facilitating users to effectively retrieve their desired products. While these websites offer a wide range of products, the existing user interface and navigation system present challenges for customers to quickly and efficiently find specific computers and gadgets, resulting in a subpar user experience.
\newline
\newline
The current user interaction design of e-commerce websites in the computer and gadgets sector lacks the necessary features and functionalities to enhance the browsing and search experience. Users often struggle to locate their desired products due to limited search filters, inadequate categorization, and a lack of intuitive navigation options. This leads to frustration, increased browsing time, and a potential loss of interest in exploring the full range of offerings.
\newline
\newline
One of the specific challenges lies in the use of filters to narrow down search results, which often requires users to possess a certain level of domain knowledge in gadgets, electronics, or computers. This can cause confusion for users who are new to this field and may not be familiar with the specific terminology or technical specifications. Similarly, searching using full-text search poses difficulties, as users need to have a good understanding of how to formulate effective search queries to retrieve products that meet their needs.
\newline
\newline
In summary, by empowering users to find products that meet their needs without requiring extensive technical expertise or struggling formalate what they needs in proper words, the websites can attract a broader customer base, drive conversion rates, and establish a competitive advantage in the market.